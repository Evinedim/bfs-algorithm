\documentclass[14pt, a4paper]{extreport}

\usepackage[utf8]{inputenc}
\usepackage[T2A]{fontenc}
\usepackage[russian]{babel}
\usepackage{pdfpages}
\usepackage{graphicx}

\begin{document}

\includepdf[pages={1}]{pages/title-page.pdf}
\includepdf[pages={1,2}]{pages/task-page.pdf}

\section*{\centering Алгоритм}

\subsection*{\normalsize Что такое граф?}

Граф — это математическая модель для отображения связей между объектами, состоящая из вершин (точек, объектов) и рёбер (линий, связей) их соединяющих.

\subsection*{\normalsize Что такое обход графа?}

Простыми словами, обход графа — это переход от одной его вершины к другой в поисках свойств связей этих вершин. Связи (линии, соединяющие вершины) называются направлениями, путями, гранями или ребрами графа. Вершины графа также именуются узлами.

\subsection*{\normalsize Поиск в ширину}

BFS (breadth-first search) следует концепции «расширяйся, поднимаясь на высоту птичьего полета» («go wide, bird’s eye-view»). Вместо того, чтобы двигаться по определенному пути до конца, BFS подразумевает посещение ближайших к s соседей за одно действие (шаг), затем посещение соседей соседей и так до тех пор, пока не будет обнаружено t. Это означает следующее:

\begin{figure}[ht]
    \centering
    \includegraphics[scale=0.32]{images/image2.png}
\end{figure}

\begin{figure}
    \centering
    \includegraphics[scale=0.32]{images/image3.png}
\end{figure}

\begin{figure}
    \centering
    \includegraphics[scale=0.32]{images/image4.png}
\end{figure}

\subsection*{\normalsize Возникает вопрос: как узнать, каких соседей следует посетить первыми?}

Для этого мы можем воспользоваться концепцией «первым вошел, первым вышел» (first-in-first-out, FIFO) из очереди (queue). Мы помещаем в очередь сначала ближайшую к нам вершину, затем ее непосещенных соседей, и продолжаем этот процесс, пока очередь не опустеет или пока мы не найдем искомую вершину.

\subsection*{\normalsize Анализ BFS}

Очередь предполагает обработку каждой вершины перед достижением пункта назначения. Это означает, что, в худшем случае, BFS исследует все вершины и грани.

\noindent Таким образом, время выполнения BFS также составляет O(V + E), а поскольку мы используем очередь, вмещающую все вершины, его пространственная сложность составляет O(V).

\section*{\centering Реализация алгоритма}

\end{document}